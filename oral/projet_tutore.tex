\documentclass[10pt, compress]{beamer}

\usetheme{m}

\usepackage{booktabs}
\usepackage{xcolor}
\usepackage[scale=2]{ccicons}

\usepgfplotslibrary{dateplot}

%Colors
\definecolor{light_gray}{HTML}{aaaaaa}

%Commands
\newcommand{\up}[1]{\textsuperscript{\textbf{\textsc{#1}}}}

\title{Conception et développement d’une application d’annotation thématique}
\subtitle{dans l’environnement Gate}
\date{\today}
\author{Charles Follet \and Rolan Bary}
\institute{Université de Pau et Pays de l'Adour}

\begin{document}

\maketitle

%%%%%%%%%%%%%%%%%%%%%%%%%%%%
%   Structure du numeraire %
%%%%%%%%%%%%%%%%%%%%%%%%%%%%
\section{Structure du numéraire}
\begin{frame}[fragile]
\frametitle{~}
\begin{scriptsize}
Agen (Lot-et-Garonne)\\~\\

Type de 793/4-812: Charlemagne (768-814)\\
Denier de Charlemagne (24 exemplaires étudiés)\\
+ CARLVS REX FR croix + AGINNO monogramme\\
Gariel XXII 26; Morrison-Grunthal 177-179; Depeyrot (1) (2) 1; Collections: Berlin 1,64, 1,45;\\
Bruxelles 1,25, 1,25; Charleville-Mézières 1,56; Copenhague 1,31; Garett 1,73; Grenoble; MEC 735 \\
(1,57), 736 (1,24); Monnaie de Paris 88 (1,55); New York 1,64, 1,57; Prou 792 (1,36), 793 (1,60), \\
794 (1,50) Trésors: Biebrich (790-814), 2 ex. (MG 11, Vôlkers, p. 182) (1,60); Ibersheim (814), 1 ex. \\
(MG 13, Vôlkers, p. 186) (1,64); Dorestad (822), 4 ex. (MG 18, Vôlkers, p. 139; H. 7) (1,35);\\
Trouvailles: Bolsward, 1 ex. (H. 528) (1,26).
\end{scriptsize}
\end{frame}


\begin{frame}[fragile]
  \frametitle{L'atelier}
  \begin{scriptsize}
\textbf{Agen (Lot-et-Garonne)}\\~\\
\textcolor{light_gray}{
Type de 793/4-812: Charlemagne (768-814)\\
Denier de Charlemagne (24 exemplaires étudiés)\\
+ CARLVS REX FR croix + AGINNO monogramme\\
Gariel XXII 26; Morrison-Grunthal 177-179; Depeyrot (1) (2) 1; Collections: Berlin 1,64, 1,45; \\
Bruxelles 1,25, 1,25; Charleville-Mézières 1,56; Copenhague 1,31; Garett 1,73; Grenoble; MEC 735 \\
(1,57), 736 (1,24); Monnaie de Paris 88 (1,55); New York 1,64, 1,57; Prou 792 (1,36), 793 (1,60), \\
794 (1,50) Trésors: Biebrich (790-814), 2 ex. (MG 11, Vôlkers, p. 182) (1,60); Ibersheim (814), 1 ex. \\
(MG 13, Vôlkers, p. 186) (1,64); Dorestad (822), 4 ex. (MG 18, Vôlkers, p. 139; H. 7) (1,35); \\Trouvailles: Bolsward, 1 ex. (H. 528) (1,26).
} 
    \end{scriptsize}
\end{frame}

\begin{frame}[fragile]
  \frametitle{Le type}
  \begin{scriptsize}
\textcolor{light_gray}{Agen (Lot-et-Garonne)}\\~\\

\textbf{Type de 793/4-812: Charlemagne (768-814)}\\
\textcolor{light_gray}{
Denier de Charlemagne (24 exemplaires étudiés)\\
+ CARLVS REX FR croix + AGINNO monogramme\\
Gariel XXII 26; Morrison-Grunthal 177-179; Depeyrot (1) (2) 1; Collections: Berlin 1,64, 1,45; \\
Bruxelles 1,25, 1,25; Charleville-Mézières 1,56; Copenhague 1,31; Garett 1,73; Grenoble; MEC 735 \\
(1,57), 736 (1,24); Monnaie de Paris 88 (1,55); New York 1,64, 1,57; Prou 792 (1,36), 793 (1,60), \\
794 (1,50) Trésors: Biebrich (790-814), 2 ex. (MG 11, Vôlkers, p. 182) (1,60); Ibersheim (814), 1 ex. \\
(MG 13, Vôlkers, p. 186) (1,64); Dorestad (822), 4 ex. (MG 18, Vôlkers, p. 139; H. 7) (1,35); \\Trouvailles: Bolsward, 1 ex. (H. 528) (1,26).
} 
    \end{scriptsize}
\end{frame}

\begin{frame}[fragile]
  \frametitle{La nature}
  \begin{scriptsize}
\textcolor{light_gray}{Agen (Lot-et-Garonne)}\\~\\

\textcolor{light_gray}{Type de 793/4-812: Charlemagne (768-814)}\\

\textbf{Denier de Charlemagne (24 exemplaires étudiés)}\\
\textcolor{light_gray}{
+ CARLVS REX FR croix + AGINNO monogramme\\
Gariel XXII 26; Morrison-Grunthal 177-179; Depeyrot (1) (2) 1; Collections: Berlin 1,64, 1,45; \\
Bruxelles 1,25, 1,25; Charleville-Mézières 1,56; Copenhague 1,31; Garett 1,73; Grenoble; MEC 735 \\
(1,57), 736 (1,24); Monnaie de Paris 88 (1,55); New York 1,64, 1,57; Prou 792 (1,36), 793 (1,60), \\
794 (1,50) Trésors: Biebrich (790-814), 2 ex. (MG 11, Vôlkers, p. 182) (1,60); Ibersheim (814), 1 ex. \\
(MG 13, Vôlkers, p. 186) (1,64); Dorestad (822), 4 ex. (MG 18, Vôlkers, p. 139; H. 7) (1,35); \\Trouvailles: Bolsward, 1 ex. (H. 528) (1,26).
} 
    \end{scriptsize}
\end{frame}

\begin{frame}[fragile]
  \frametitle{La légende}
  \begin{scriptsize}
\textcolor{light_gray}{Agen (Lot-et-Garonne)}\\~\\

\textcolor{light_gray}{Type de 793/4-812: Charlemagne (768-814)\\
Denier de Charlemagne (24 exemplaires étudiés)}\\

\textbf{+ CARLVS REX FR croix + AGINNO monogramme}\\
\textcolor{light_gray}{
Gariel XXII 26; Morrison-Grunthal 177-179; Depeyrot (1) (2) 1; Collections: Berlin 1,64, 1,45; \\
Bruxelles 1,25, 1,25; Charleville-Mézières 1,56; Copenhague 1,31; Garett 1,73; Grenoble; MEC 735 \\
(1,57), 736 (1,24); Monnaie de Paris 88 (1,55); New York 1,64, 1,57; Prou 792 (1,36), 793 (1,60), \\
794 (1,50) Trésors: Biebrich (790-814), 2 ex. (MG 11, Vôlkers, p. 182) (1,60); Ibersheim (814), 1 ex. \\
(MG 13, Vôlkers, p. 186) (1,64); Dorestad (822), 4 ex. (MG 18, Vôlkers, p. 139; H. 7) (1,35); \\
Trouvailles: Bolsward, 1 ex. (H. 528) (1,26).
} 
    \end{scriptsize}
\end{frame}

\begin{frame}[fragile]
  \frametitle{Les collections}
  \begin{scriptsize}
\textcolor{light_gray}{Agen (Lot-et-Garonne)}\\~\\

\textcolor{light_gray}{
Type de 793/4-812: Charlemagne (768-814)\\
Denier de Charlemagne (24 exemplaires étudiés)\\
+ CARLVS REX FR croix + AGINNO monogramme
}\\
\textcolor{light_gray}{
Gariel XXII 26; Morrison-Grunthal 177-179; Depeyrot (1) (2) 1; }\textbf{Collections: Berlin 1,64, 1,45; \\
Bruxelles 1,25, 1,25; Charleville-Mézières 1,56; Copenhague 1,31; Garett 1,73; Grenoble; MEC 735 \\
(1,57), 736 (1,24); Monnaie de Paris 88 (1,55); New York 1,64, 1,57; Prou 792 (1,36), 793 (1,60), \\
794 (1,50) }\textcolor{light_gray}{Trésors: Biebrich (790-814), 2 ex. (MG 11, Vôlkers, p. 182) (1,60); Ibersheim (814), 1 ex. \\(MG 13, Vôlkers, p. 186) (1,64); Dorestad (822), 4 ex. (MG 18, Vôlkers, p. 139; H. 7) (1,35); \\Trouvailles: Bolsward, 1 ex. (H. 528) (1,26).
} 
    \end{scriptsize}
\end{frame}

\begin{frame}[fragile]
  \frametitle{Les trésors}
  \begin{scriptsize}
\textcolor{light_gray}{Agen (Lot-et-Garonne)}\\~\\

\textcolor{light_gray}{
Type de 793/4-812: Charlemagne (768-814)\\
Denier de Charlemagne (24 exemplaires étudiés)\\
+ CARLVS REX FR croix + AGINNO monogramme
}\\
\textcolor{light_gray}{
Gariel XXII 26; Morrison-Grunthal 177-179; Depeyrot (1) (2) 1;
Collections: Berlin 1,64, 1,45; \\
Bruxelles 1,25, 1,25; Charleville-Mézières 1,56; Copenhague 1,31; Garett 1,73; Grenoble; MEC 735 \\
(1,57), 736 (1,24); Monnaie de Paris 88 (1,55); New York 1,64, 1,57; Prou 792 (1,36), 793 (1,60), \\
794 (1,50) }\textbf{Trésors: Biebrich (790-814), 2 ex. (MG 11, Vôlkers, p. 182) (1,60); Ibersheim (814), 1 ex. \\
(MG 13, Vôlkers, p. 186) (1,64); Dorestad (822), 4 ex. (MG 18, Vôlkers, p. 139; H. 7) (1,35); }\\
\textcolor{light_gray}{
Trouvailles: Bolsward, 1 ex. (H. 528) (1,26).
} 
    \end{scriptsize}
\end{frame}

\begin{frame}[fragile]
  \frametitle{Les trouvailles}
  \begin{scriptsize}
\textcolor{light_gray}{Agen (Lot-et-Garonne)}\\~\\

\textcolor{light_gray}{
Type de 793/4-812: Charlemagne (768-814)\\
Denier de Charlemagne (24 exemplaires étudiés)\\
+ CARLVS REX FR croix + AGINNO monogramme
}\\
\textcolor{light_gray}{
Gariel XXII 26; Morrison-Grunthal 177-179; Depeyrot (1) (2) 1;
Collections: Berlin 1,64, 1,45; \\
Bruxelles 1,25, 1,25; Charleville-Mézières 1,56; Copenhague 1,31; Garett 1,73; Grenoble; MEC 735 \\
(1,57), 736 (1,24); Monnaie de Paris 88 (1,55); New York 1,64, 1,57; Prou 792 (1,36), 793 (1,60), \\
794 (1,50) Trésors: Biebrich (790-814), 2 ex. (MG 11, Vôlkers, p. 182) (1,60); Ibersheim (814), 1 ex. \\
(MG 13, Vôlkers, p. 186) (1,64); Dorestad (822), 4 ex. (MG 18, Vôlkers, p. 139; H. 7) (1,35); }\\
\textbf{Trouvailles: Bolsward, 1 ex. (H. 528) (1,26).}
    \end{scriptsize}
\end{frame}

\begin{frame}[fragile]
  \frametitle{Résumé}
  \begin{scriptsize}
Agen (Lot-et-Garonne)\alert{\up{Atelier}}\\~\\
Type de 793/4-812: Charlemagne (768-814)\alert{\up{Type}}\\
Denier de Charlemagne (24 exemplaires étudiés)\alert{\up{Nature}}\\
+ CARLVS REX FR croix + AGINNO monogramme\alert{\up{Légende}}\\
Gariel XXII 26; Morrison-Grunthal 177-179; Depeyrot (1) (2) 1;
Collections: Berlin 1,64, 1,45; \\
Bruxelles 1,25, 1,25; Charleville-Mézières 1,56; Copenhague 1,31; Garett 1,73; Grenoble; MEC 735 \\
(1,57), 736 (1,24); Monnaie de Paris 88 (1,55); New York 1,64, 1,57; Prou 792 (1,36), 793 (1,60), \\
794 (1,50)\alert{\up{Collections}} Trésors: Biebrich (790-814), 2 ex. (MG 11, Vôlkers, p. 182) (1,60); Ibersheim (814), 1 ex. \\
(MG 13, Vôlkers, p. 186) (1,64); Dorestad (822), 4 ex. (MG 18, Vôlkers, p. 139; H. 7) (1,35);\alert{\up{Trésors}} \\
Trouvailles: Bolsward, 1 ex. (H. 528) (1,26)\alert{\up{Trouvailles}}.
    \end{scriptsize}
\end{frame}


\section{étapes de développement}




\end{document}