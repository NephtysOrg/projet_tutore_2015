\documentclass[a4paper, 10pt]{report}

%% Packages
\usepackage[utf8]{inputenc}
\usepackage[francais]{babel}
\usepackage{framed}
\usepackage[usenames,dvipsnames]{xcolor}
\usepackage{amssymb}
\usepackage{sectsty}
\usepackage{multicol}
\usepackage{enumitem}

%% Couleurs
\definecolor{vert-0}{HTML}{459436}
\definecolor{vert-1}{HTML}{93CC4A}
\definecolor{bleu-0}{HTML}{005572}
\definecolor{bleu-1}{HTML}{006573}
\definecolor{bleu-2}{HTML}{008B8D}
\colorlet{shadecolor}{gray!7}

%% Parametres
\setcounter{tocdepth}{2}
\sectionfont{\color{bleu-1}}
\subsectionfont{\color{bleu-2}}

%% Commandes
\renewcommand{\thesection}{\arabic{section}} % Les sections n'affichent pas le numero du chapitre
\newcommand{\keyword}[1]{\noindent\textbf{\underline{#1}}} % Mot clef des algos en gras et soulignés
\setlist[itemize]{label=\tiny$\bullet$}

%% Environements
\renewenvironment{shaded}
	{
  	\def\FrameCommand{\fboxsep=\FrameSep \colorbox{shadecolor}
  	}
  	\MakeFramed{\advance\hsize-\width \FrameRestore\FrameRestore}}
 {\endMakeFramed}


\newenvironment{discussion}
    {
    \noindent\textbf{\textcolor{vert-1}{$\blacksquare$  Discussion : \\}}
    }
    {
    \noindent\textcolor{vert-1}{\\$\blacksquare$}\\
    }
    
\newenvironment{tache}[1][]
    {
    \noindent\textbf{\textcolor{vert-0}{$\clubsuit$  Tâches à réaliser pour le #1 : }}
    }
    {
    \noindent\textcolor{vert-0}{\\$\clubsuit$}\\
    }
    
\newenvironment{ordre}[1][]
    {
    \noindent\textbf{\textcolor{vert-0}{$\clubsuit$  Ordre du jour du #1 : }}
    }
    {
    }

\begin{document}
%% Couverture
\title{{\huge \textbf{Compte rendu de réunion et Ordres du jour \\ Projet tutoré 2015}}}
\author{Roland Bary, Charles Follet}
\date{Encadrants : Marie-Noelle Bessagnet, Annig Lacayrelle,  Albert Royer, Christian Sallaberry}
\maketitle
\tableofcontents
\newpage

\section{Jeudi 5 Février}
\begin{discussion}
\begin{itemize}
\item Rencontre entre l'équipe d'encadrants et les groupes associés aux projets.
\item Présentation brève des cahiers de charges relatifs au différents projets et de l'environnement d'annotation thématique GATE.
\end{itemize}
\end{discussion}

\begin{tache}[09/02/2015]
\begin{itemize}
\item Installer l'environnement GATE et effectuer quelques tests pour se familiariser avec.
\item Installer  le logiciel TreeTagger et effectuer quelques tests pour se familiariser avec.
\item Lire la documentation de GATE jusqu'à la partie 6.
\item Lire des ressources documentaires de chaque projet.
\end{itemize}
\end{tache} 

\section{Lundi 9 Février 2015}
\subsection{Paramétrage de GATE}
Convention de nommage :\\
GATEdir : Chemin d'installation de GATE.\\
TREETAGGERdir : Chemin d'installation de TreeTagger.
\subsubsection{ANNIE}
\begin{itemize}
\item Lancer le CREOLE plugin manager en cliquant sur le logo CREOLE (puzzle).
\item Cocher la case Load Now et Load Always du plugin ANNIE.
\item Cliquer sur Apply All.
\item Fermer le plugin manager.
\end{itemize}
\subsubsection{Unicode Tokenizer}
Dans GATE :
\begin{itemize}
\item Clic droit sur Processing Ressource dans l'arborescence de gauche.
\item New  GATE Unicode Tokenizer avec encoding : utf8, rulesURL :\begin{verbatim}
GATEdir/plugins/Lang_French/tokeniser/FrenchTokeniser.rules
\end{verbatim}
\end{itemize}
\subsubsection{Inclusion TreeTagger}
\begin{itemize}
\item Éditer 
\begin{verbatim}
GATEdir/plugin/Tagger_Framework/resources/TreeTagger/treetagger-french-gate
\end{verbatim}
\item Ajouter/modifier les lignes suivantes 
\begin{verbatim}
 BIN=TREETAGGERdir/bin
 LIB=TREETAGGERdir/lib
 CMD=TREETAGGERdir/cmd
\end{verbatim}
\end{itemize}
\subsubsection{Récupérer une application}
\begin{itemize}
\item Clic droit sur Applications puis Restore application from file...
\item sélectionner \begin{verbatim}
GATEdir/plugins/Lang_French/french+tagger.gapp
\end{verbatim}
\item Dans French\_NE : clic gauche sur Treetager-FR-NoTokenization
\item modifier encoding: utf-8 et taggerBinary : \begin{verbatim}
GATEdir/plugins/Tagger_Framework/resources/TreeTagger/tree-tagger-french-gate
\end{verbatim}
\end{itemize}
\subsection{Corrections de bugs}
\subsubsection{GAWK}
installer gawk : \begin{verbatim}sudo apt-get install gawk\end{verbatim}
\subsubsection{Renommage}
Copie de french-utf8.par en french.par dans le dossier \begin{verbatim} TREETAGGERdir/lib\end{verbatim}
\subsubsection{LuckyException}
Dans \begin{verbatim}GATEdir/plugin/Lang_French/tokeniser/postprocess.jap\end{verbatim}
\newpage
\section{Mardi 03 Mars 2015}
\begin{discussion}
Il a été au cours de cette réunion, de présenter le travail que nous avons effectué durant les vacances du mois de février :
\begin{itemize}
\item Définition des éléments d'information ciblés, par une ébauche de construction de règles d'annotations. Les éléments ciblés étant les suivants :
      \begin{itemize}
	\item Spatiale: villes, départements, villes de pays frontaliers
	\item Temporelle: période d'émission des monnaies
	\item Thématique: nature de la monnaie, légende du droit et légende du revers, le type de la monnaie, le nom de l'atelier, personnages associés aux monnaies
      \end{itemize}
\item Construction d'un premier Gazetter à partir des nom de villes et départements de France récupérés sur le site web de L'INSEE
\item Tests d'annotation sur un fichier d'extension .txt (Tiré de la ressource documentaire) avec le dit gazetter constitué.
\end{itemize}
Aussi plusieurs remarques ont été mises en évidence, à savoir :
\begin{itemize}
 \item L'amélioration de l'ébauche de construction des règles d'annotation qui a été présentée, par l'ajout des thématiques 'collection','Trésors' et 'trouvailles' et 
 \item Le traitement en priorité cas généraux d'annotation et noter les cas spécifiques rencontrés
 \item Formaliser aux mieux l'information, à l'aide de notions bien connues (i.e Expression Régulières, automates, ...)
 \item Constitution de Gazetter à partir d'informations directement extraites de la ressource documentaire
\end{itemize}
\end{discussion}
%%%%%%%%%%% Taches a realiser %%%%%%%%%%%%%%%%%%%%%%%%%%%
\begin{tache}
A l'issue de cette réunion il nous a été recommandé de réaliser les tâches suivantes pour la prochaine rencontre:
\begin{itemize}
 \item Scan et Océrisation des pages constituant les chapitres I à III, afin d'extraire des informations nécessaires à la constructions de Gazzeters
 \item Améliorer l'ébauche de construction des règles d'annotation, par des illustrations, et des notes de problèmes éventuellement rencontrés
 \item Être capable de retrouver les noms des ateliers, les périodes, les collections, trésors, trouvailles...
\end{itemize}
\end{tache}
%%%%%%%%%%% Ordre du jour %%%%%%%%%%%%%%%%%%%%%%%%%%%%%%%
\begin{ordre}[]
\end{ordre}
\begin{itemize}
 \item Présentation
 \item
 \item 
\end{itemize}

\end{document}